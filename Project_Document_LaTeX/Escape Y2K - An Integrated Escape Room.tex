\documentclass[conference]{IEEEtran}
\IEEEoverridecommandlockouts
\usepackage{cite}
\usepackage{amsmath,amssymb,amsfonts}
\usepackage{algorithmic}
\usepackage{graphicx}
\usepackage{textcomp}
\usepackage{xcolor}
\usepackage{url}
\usepackage{wrapfig}
\usepackage{float}
\def\BibTeX{{\rm B\kern-.05em{\sc i\kern-.025em b}\kern-.08em
    T\kern-.1667em\lower.7ex\hbox{E}\kern-.125emX}}
\begin{document}

\title{Escape Y2K - An Integrated Escape Room}

\author{Kyle L. Sedgwick, Jake D. Bales, Nami Eskandarian}

\author{\IEEEauthorblockN{1\textsuperscript{st} Kyle L. Sedgwick}
    \IEEEauthorblockA{\textit{College of Engineering} \\
        \textit{University of Utah}\\
        Salt Lake City, U.S \\}
    \and
    \IEEEauthorblockN{2\textsuperscript{nd} Jake D. Bales}
    \IEEEauthorblockA{\textit{College of Engineering} \\
        \textit{University of Utah}\\
        Salt Lake City, U.S \\}
    \and
    \IEEEauthorblockN{3\textsuperscript{rd} Nami Eskandarian}
    \IEEEauthorblockA{\textit{College of Engineering} \\
        \textit{University of Utah}\\
        Salt Lake City, U.S \\}
}



\maketitle

\begin{abstract}
    “ESCAPE Y2K” is an interactive escape room experience that relies on computer
engineering as its main control source. The escape room is built to be an autonomous,
immersive, sci-fi, horror experience. A variety of sensors will be used to accomplish this
including a sonar range finder, noise sensor, pressure and heat sensors, nfc/rfid, and
movement sensors. Other technology to be implemented includes image/audio processing,
bluetooth, and digital/analog circuit design. A stretch goal for this project is to make this
room modular and portable, allowing it to be set up in any place and with any size of room.\\\\
One of the major themes of the escape room is time travel. The experience will run
on a clock that ticks between 1:00 PM and 12:00 (midnight) where certain events are
dependent on the time. This can include cabinets opening during a specific time interval or
locks having different passcode combinations depending on the hour hand. Players in the
room are able to rewind or forward the time however they wish based on the minimum and
maximum the time can go. Past 8:00, the game will transition to a nighttime mode where
fake windows in the room will shine a light to simulate a creature looking inside. If a player is
caught in this light, the room enters a danger state and the team will incur a penalty to the
amount of time they have to escape. This penalty will also occur if the clock reaches 12:00.
The game will conclude either when the players all exit the room safely, or the game clock
expires.\\\\
The maximum amount of time players will have to escape will be 30 minutes,
however, this may lessen if the player clock reaches midnight or if any of the players are
seen by the monster. Each puzzle will take between 3 and 5 minutes to solve, allowing for a
maximum of 7 or 8 (maybe) puzzles. Another stretch goal is to create a pool of puzzles from
which the room can draw, giving each group that comes into the room a unique escape
experience. These puzzles will be hard, but will still be easy enough to actually allow players
to enjoy the experience and not be frustrated by the difficulty level of the puzzles.\\
\end{abstract}

\begin{IEEEkeywords}
    Analog, Embedded Systems, Escape Room, Horror, Interactive, Networking, Science Fiction
\end{IEEEkeywords}

\section{Introduction}
Introduction

\section{Section 1}
Section 1

\section{Section 2}
Section 2
\subsection{Subsection 1}
Subsection 1
\subsection{Subsection 2 }
Subsection 2

\section{Conclusion}
Conclusion


\end{document}
