\documentclass[conference]{IEEEtran}
\IEEEoverridecommandlockouts
\usepackage{cite}
\usepackage{amsmath,amssymb,amsfonts}
\usepackage{algorithmic}
\usepackage{graphicx}
\usepackage{textcomp}
\usepackage{xcolor}
\usepackage{url}
\usepackage{wrapfig}
\usepackage{float}
\def\BibTeX{{\rm B\kern-.05em{\sc i\kern-.025em b}\kern-.08em
    T\kern-.1667em\lower.7ex\hbox{E}\kern-.125emX}}
\begin{document}

\title{ESCAPE Y2K - An Integrated Escape Room}

\author{Kyle L. Sedgwick, Jake D. Bales, Nami Eskandarian}

\author{\IEEEauthorblockN{1\textsuperscript{st} Kyle L. Sedgwick}
    \IEEEauthorblockA{\textit{College of Engineering} \\
        \textit{University of Utah}\\
        Salt Lake City, U.S \\}
    \and
    \IEEEauthorblockN{2\textsuperscript{nd} Jake D. Bales}
    \IEEEauthorblockA{\textit{College of Engineering} \\
        \textit{University of Utah}\\
        Salt Lake City, U.S \\}
    \and
    \IEEEauthorblockN{3\textsuperscript{rd} Nami Eskandarian}
    \IEEEauthorblockA{\textit{College of Engineering} \\
        \textit{University of Utah}\\
        Salt Lake City, U.S \\}
}



\maketitle

\begin{abstract}
    “ESCAPE Y2K” is an interactive escape room experience that relies on computer
    engineering as its main control source. The escape room is built to be an autonomous,
    immersive, sci-fi, horror experience. A variety of sensors will be used to accomplish this
    including a sonar range finder, noise sensor, pressure and heat sensors, nfc/rfid, and
    movement sensors. Other technology to be implemented includes image/audio processing,
    bluetooth, and digital/analog circuit design. A stretch goal for this project is to make this
    room modular and portable, allowing it to be set up in any place and with any size of room.\\\\
    One of the major themes of the escape room is time travel. The experience will run
    on a clock that ticks between 1:00 PM and 12:00 (midnight) where certain events are
    dependent on the time. This can include cabinets opening during a specific time interval or
    locks having different passcode combinations depending on the hour hand. Players in the
    room are able to rewind or forward the time however they wish based on the minimum and
    maximum the time can go. Past 8:00, the game will transition to a nighttime mode where
    fake windows in the room will shine a light to simulate a creature looking inside. If a player is
    caught in this light, the room enters a danger state and the team will incur a penalty to the
    amount of time they have to escape. This penalty will also occur if the clock reaches 12:00.
    The game will conclude either when the players all exit the room safely, or the game clock
    expires.\\\\
    The maximum amount of time players will have to escape will be 30 minutes,
    however, this may lessen if the player clock reaches midnight or if any of the players are
    seen by the monster. Each puzzle will take between 3 and 5 minutes to solve, allowing for a
    maximum of 7 or 8 (maybe) puzzles. Another stretch goal is to create a pool of puzzles from
    which the room can draw, giving each group that comes into the room a unique escape
    experience. These puzzles will be hard, but will still be easy enough to actually allow players
    to enjoy the experience and not be frustrated by the difficulty level of the puzzles.\\
\end{abstract}

\begin{IEEEkeywords}
    Analog, Embedded Systems, Escape Room, Horror, Interactive, Networking, Science Fiction
\end{IEEEkeywords}

\section{Introduction}
In general, innovation in the escape room industry is minimal; if you've been to one or two you've seen
how pretty much any of them are going to work. The level of difficulty from room to room may vary, and some
of the puzzles could be interesting, but there haven't been any groundbreaking changes made to the scene since
its inception. Our goal is to THIS ISNT DONE YET, JAKE STARTED IT BUT YOU CAN TOTALLY DO WHATEVER YOU WANT WITH IT.

\section{Our Vision}
Our vision for this project is to take a unique spin on the formula that is most commonly used in escape rooms.

Make this 3 paragraphs. Cover 1. Background on escape rooms 2. Why we wanna do this 3. How were going to
incorporate computer engineering into it

\subsection*{What is an escape room?}
If you arent very familiar with escape rooms, the basic idea is to provide people with an interactive
and exciting puzzle experience. Players start by being "locked" in a room (you're never actually locked
in, for safety reasons) with a set of instructions that lead them through a series of puzzles. Some of
these puzzles are more traditional, such as solving a cypher or figuring out a combination for a lock,
while others make the players think a little bit deeper. Many of these puzzles are on the simple size in
an attempt to have a good balance of fun and difficulty. And, many of these rooms attempt to fit their
puzzles within a certain theme, such as escaping from an Egyptian tomb or trying to escape from the zombie
apcalypse.

\subsection*{Our motive behind the project}


\subsection*{What makes our escape room unique?}
This section is where we should talk about the clock and how that

\section{Puzzles in our Escape Room}
This section contains a list of all of the puzzles that our escape room will feature, as well as
the solution to each of them. If you haven't already experienced the escape room, be warned that
this section does contain spoilers and will prevent you from experiencing the joy of solving the
puzzles on your own.

\subsection{Chess Board Puzzle}

\subsection{Bust}




\section{Materials Needed}


\section{Conclusion}
Conclusion


\end{document}
